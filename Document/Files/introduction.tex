\section{Introducción}

Uno de los problemas que se presentan en las sociedades humanas, más en especifico en la sociedad mexicana es el dificil acceso a los servicios públicos como salud, educación o justicia. Causando condiciones de vulnerabilidad e inseguridad. Una consecuencia de esta precariedad es la disminución en la calidad de vida de los habitantes. Con este problema surgen instituciones nacionales e internacionales para estudiar el caso, recabar información y generar indicadores que ayuden a identificar los casos en una región extensa. El consejo Nacional de Población (CONAPO) calcula el índice de marginación (IM) basado en el fundamento en el enfoque de necesidades básicas insatisfechas, utilizando el análisis de componentes principales \cite{conapo_2021}. De una manera semejante, el Consejo Nacional de Evaluación de la Política de Desarrollo Social (CONEVAL) estima el índice de rezago social (IRS) por medio de una suma ponderada de porcentajes \cite{CONEVAL_2007}. El IM utiliza como indicadores\cite{CONAPO_1995} los siguientes porcentajes:

\begin{itemize}
    \item Población analfabeta de 15 años o más.
    \item Población de 15 años o más sin educación básica.
    \item Viviendas particulares sin agua entubada.
    \item Viviendas particulares sin drenaje ni excusado.
    \item Viviendas particulares sin energía eléctrica.
    \item Viviendas particulares con piso de tierra.
    \item Viviendas particulares con hacinamiento
    \item Porcentaje de población que vive en localidades menores a 5000 habitantes.
    \item Población ocupada con ingresos de hasta 2 salarios mínimos.
\end{itemize}

Cada indicador es normalización usando la ecuación \ref{eq:standar_variable}.

\begin{equation}
    Z_{ij} = \frac{X_{ij}-\mu_j}{\sigma_j} \label{eq:standar_variable}
\end{equation}

Donde $X_{ij}$ representa al i-ésimo valor de la variable $k$, $\mu_k$ y $\sigma_k$ es el promedio y desviación estandar de la variable $k$ respectivamente. Debido a la normalización de cada indicador, el IM no puede utilizarse para realizar un seguimiento temporal. Para representar a los nueva indicadores en uno solo se aplica el análisis de componentes principales. La CONAPO emplea el método de Dalenius Hodges\cite{Dalenius_1959} para agrupar los puntajes en cinco categorias de marginación (muy alta, alta, media, baja y muy baja). El IM permite ordenar y clasificar según su categoria de marginación en el año que se calcula.

El IM se utiliza para localizar las zonas de atención prioritaria. Estas zonas tienen acceso a fondos y programas gubernamentales para el combate de la marginación y el desarrollo de la población \cite{DOF_2011,DOF_2012,DOF_2013,DOF_2014,DOF_2015,DOF_2016,DOF_2017,DOF_2018,DOF_2019}. Los programas que usan al IM para determinar las localidades que seran beneficiarias son:

\begin{itemize}
    \item Estrategia 100x100\cite{CONEVAL_2013}.
    \item Estrategia del Gobierno Federal para la Dotación de Piso Firme\cite{DOF_2020}.
    \item Programa de Apoyo Alimentario en Zonas de Atención Prioritaria\cite{SEDESOL_2008}.
    \item Programa de Apoyo a Zonas de Atención Prioritaria\cite{DOF_2014_zonas}.
\end{itemize}


\section{Introducción}

Uno de los problemas que se presentan en las sociedades humanas, más en especifico en la sociedad mexicana es el dificil acceso a los servicios públicos como salud, educación o justicia. Causando condiciones de vulnerabilidad e inseguridad. Una consecuencia de esta precariedad es la disminución en la calidad de vida de los habitantes. Con este problema surgen instituciones nacionales e internacionales para estudiar el caso, recabar información y generar indicadores que ayuden a identificar los casos en una región extensa. El consejo Nacional de Población (CONAPO) calcula el índice de marginación (IM) basado en el fundamento en el enfoque de necesidades básicas insatisfechas, utilizando el análisis de componentes principales \cite{conapo_2021}. De una manera semejante, el Consejo Nacional de Evaluación de la Política de Desarrollo Social (CONEVAL) estima el índice de rezago social (IRS) por medio de una suma ponderada de porcentajes \cite{CONEVAL_2007}. El índice de marginación se utiliza para localizar las zonas de atención prioritaria. Estas zonas tienen acceso a fondos y programas gubernamentales para el combate de la marginación y el desarrollo de la población \cite{DOF_2011,DOF_2012,DOF_2013,DOF_2014,DOF_2015,DOF_2016,DOF_2017,DOF_2018,DOF_2019}.


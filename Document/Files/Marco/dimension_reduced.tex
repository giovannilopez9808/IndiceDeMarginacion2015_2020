\subsection{Algoritmos de reducción de dimensionalidad \label{sec:methods}}

La reducción de dimensionalidad es un proceso que disminuye el tamaño de variables o que describen a un conjunto de datos. Las ventajas de la reducción de dimensionalidad son las siguientes:

\begin{itemize}
    \item Reduce el espacio de tiempo y almacenamiento requerido.
    \item La eliminación de multicolinealidad mejora el rendimiento del modelo de aprendizaje automático.
    \item Se hace más fácil de visualizar los datos cuando se reduce dos o tres dimensiones.
\end{itemize}

Los algoritmos que se implemetaron en este trabajo fueron ISOMAP, K-means, Self-Organizing-Map (SOM), T-SNE y análisis de componentes principales (PCA).

\subsubsection{ISOMAP}

Se aplico el algoritmo ISOMAP descrito en la sección \ref{sec:isomap_theory} con los datos de índice de marginación para los años 2015 y 2020. En la figura \ref{fig:isompa_2d} se observa los resultados del algoritmo para el caso bidimensional para 8, 10, 12 y 14 vecinos.

\begin{figure}[H]
    \centering
    \begin{subfigure}{8.4cm}
        \includegraphics[width=1\linewidth]{Graphics/Data_2015/ISOMAP_2D.png}
        \caption{Datos 2015}
    \end{subfigure}
    \begin{subfigure}{8.4cm}
        \includegraphics[width=1\linewidth]{Graphics/Data_2020/ISOMAP_2D.png}
        \caption{Datos 2020}
    \end{subfigure}
    \caption{Resultados de aplicar ISOMAP para el caso bidimensional para los datos de índice de marginación de los años 2015 y 2020.}
    \label{fig:isompa_2d}
\end{figure}

En la tabla \ref{table:isomap_results} se pueden descargar las visualizaciones tridimensionales de los resultados del algoritmo ISOMAP en el caso tridimensional para 8, 10, 12 y 14 vecinos para los años 2015 y 2020.

\begin{table}[H]
    \centering
    \begin{tabular}{lrr} \hline
        \multirow{2}{*}{Vecinos} & \multicolumn{2}{c}{Años}                                                                                                                                                                                                                                                \\ \cline{2-3}
                                 & 2015                                                                                                                               & 2020                                                                                                                               \\ \hline
        8                        & \href{https://github.com/giovannilopez9808/Reconocimiento_de_patrones_proyecto/raw/main/Graphics/Data_2015/ISOMAP_3D_8.mp4}{Link}  & \href{https://github.com/giovannilopez9808/Reconocimiento_de_patrones_proyecto/raw/main/Graphics/Data_2020/ISOMAP_3D_8.mp4}{Link}  \\
        10                       & \href{https://github.com/giovannilopez9808/Reconocimiento_de_patrones_proyecto/raw/main/Graphics/Data_2015/ISOMAP_3D_10.mp4}{Link} & \href{https://github.com/giovannilopez9808/Reconocimiento_de_patrones_proyecto/raw/main/Graphics/Data_2020/ISOMAP_3D_10.mp4}{Link} \\
        12                       & \href{https://github.com/giovannilopez9808/Reconocimiento_de_patrones_proyecto/raw/main/Graphics/Data_2015/ISOMAP_3D_12.mp4}{Link} & \href{https://github.com/giovannilopez9808/Reconocimiento_de_patrones_proyecto/raw/main/Graphics/Data_2020/ISOMAP_3D_12.mp4}{Link} \\
        14                       & \href{https://github.com/giovannilopez9808/Reconocimiento_de_patrones_proyecto/raw/main/Graphics/Data_2015/ISOMAP_3D_14.mp4}{Link} & \href{https://github.com/giovannilopez9808/Reconocimiento_de_patrones_proyecto/raw/main/Graphics/Data_2020/ISOMAP_3D_14.mp4}{Link} \\ \hline
    \end{tabular}
    \caption{Link de descarga para las visualizaciones tridimensionales de los resultados de ISOMAP para 8, 10, 12 y 14 vecinos en los años 2015 y 2020.}
    \label{table:isomap_results}
\end{table}

\subsubsection{K-means}

En la figura \ref{fig:kmeans} se muestran las tablas de confusión resultantes de aplicar el algoritmo de K-means descrito en la sección \ref{sec:kmeans} usando el inicializador k-means++ y uno aleatorio.

\begin{figure}[H]
    \centering
    \begin{subfigure}{8.4cm}
        \includegraphics[width=1\linewidth]{Graphics/Data_2015/Kmeans++_confusion_matrix.png}
        \includegraphics[width=1\linewidth]{Graphics/Data_2015/Kmeans_random_confusion_matrix.png}
        \caption{Datos 2015}
    \end{subfigure}
    \begin{subfigure}{8.4cm}
        \includegraphics[width=1\linewidth]{Graphics/Data_2020/Kmeans++_confusion_matrix.png}
        \includegraphics[width=1\linewidth]{Graphics/Data_2020/Kmeans_random_confusion_matrix.png}
        \caption{Datos 2020}
    \end{subfigure}
    \caption{Matrices de confusión resultantes del algoritmo K-means para los datos de índice de marginación de los años 2015 y 2020.}
    \label{fig:kmeans}
\end{figure}

\subsubsection{Self-Organizing-Map}

Self-Organizing-Map (SOM) es un algoritmo de clasificación no supervisada. Su objetivo es reducir la dimensionalidad del los datos dados preservando la estructura topológica. El algoritmo minimiza la ecuación \ref{eq:som}.

\begin{equation}
    C(\{m_l\},\{l(i)\}) = \sum_i \sum_k h(||c_k-c_{l(i)}||^2) ||x_i-m_k||^2
    \label{eq:som}
\end{equation}

Donde $m_l$ es el representante más cercano a $x_i$, $h()$ es una función decreciente y positiva y $c_k$ son las celdas donde se asociara el dado $x_i$.

\subsubsection{T-SNE \label{sec:tsne}}

El algoritmo de T-SNE consiste en crear una distribución de probabilidad que representante las similitudes entre vecinos en un espacio de gran dimensión en un espacio de menor dimensión. Para cada elemento del conjunto de datos se centra una distribución gaussiana alrededor del elemento. En seguida se obtiene la densidad bajo la distribución y normalizamos el valor, calculando así una probabilidad condicional (ecuación \ref{eq:tsne_p_probability}).

\begin{equation}
    P_i=P_{j|i} = \frac{exp(-||x_i-x_j||^2/\sigma)}{\sum\limits_{k\neq i} exp(-||x_k-x_i||^2/\sigma)} \label{eq:tsne_p_probability}
\end{equation}

El valor de $\sigma$ se define por medio de un parámetro llamado perplejidad, el cual corresponde al número de vecinos alredeor de cada punto. A una mayor probabilidad condicional, los elementos ij son más similares. A partir de las probabilidades condicionales se tratan encontrar pares de elementos tales que las distribuciones $P_{i|j}$ y $q_{i|j}$ se parecen. El valor de $q_{i|j}$ se obtiene mediante la distribución T (ecuación \ref{eq:tsne_q_probability}).

\begin{equation}
    Q_i=q_{j|i} = \frac{exp(-||x_i-x_j||^2)}{\sum\limits_{k\neq i} exp(-||x_k-x_i||^2)} \label{eq:tsne_q_probability}
\end{equation}

Por lo tanto, el algoritmo T-SNE se centra en minimizar la distancia Kullback-Leiber (ecuación \ref{eq:tsne_cost_function})

\begin{equation}
    min\;\; d_i (P_i;Q_i) \qquad d(P^1;P^2) = \sum_i P_i^1 \log \left(\frac{P^1_i}{P^2_i} \right)
    \label{eq:tsne_cost_function}
\end{equation}

\subsection{Análisis de componentes principales}

Se aplico el algoritmo de PCA descrito en la sección \ref{sec:pca} con el kernel lineal, polinomial, gaussiano y sigmoide y los parámetros de la tabla \ref{table:pca_parameters}. En la figura \ref{fig:PCA_2d} se observan los resultados para el caso bidimensional.

\begin{figure}[H]
    \centering
    \begin{subfigure}{8.4cm}
        \includegraphics[width=1\linewidth]{Graphics/Data_2015/PCA_2D.png}
        \caption{Datos 2015}
    \end{subfigure}
    \begin{subfigure}{8.4cm}
        \includegraphics[width=1\linewidth]{Graphics/Data_2020/PCA_2D.png}
        \caption{Datos 2020}
    \end{subfigure}
    \caption{Resultados de aplicar PCA para el caso bidimensional para los datos de índice de marginación de los años 2015 y 2020.}
    \label{fig:PCA_2d}
\end{figure}

En la tabla \ref{table:pca_results} se pueden descargar las visualizaciones tridimensionales de los resultados del algoritmo PCA en el caso tridimensional.

\begin{table}[H]
    \centering
    \begin{tabular}{lrr} \hline
        \multirow{2}{*}{Kernel} & \multicolumn{2}{c}{Años}                                                                                                                                                                                                                                                    \\ \cline{2-3}
                                & 2015                                                                                                                                 & 2020                                                                                                                                 \\ \hline
        Lineal                  & \href{https://github.com/giovannilopez9808/Reconocimiento_de_patrones_proyecto/raw/main/Graphics/Data_2015/PCA_3D_linear.mp4}{Link}  & \href{https://github.com/giovannilopez9808/Reconocimiento_de_patrones_proyecto/raw/main/Graphics/Data_2020/PCA_3D_linear.mp4}{Link}  \\
        Coseno                  & \href{https://github.com/giovannilopez9808/Reconocimiento_de_patrones_proyecto/raw/main/Graphics/Data_2015/PCA_3D_cosine.mp4}{Link}  & \href{https://github.com/giovannilopez9808/Reconocimiento_de_patrones_proyecto/raw/main/Graphics/Data_2020/PCA_3D_cosine.mp4}{Link}  \\
        Gaussiano               & \href{https://github.com/giovannilopez9808/Reconocimiento_de_patrones_proyecto/raw/main/Graphics/Data_2015/PCA_3D_rbf.mp4}{Link}     & \href{https://github.com/giovannilopez9808/Reconocimiento_de_patrones_proyecto/raw/main/Graphics/Data_2020/PCA_3D_rbf.mp4}{Link}     \\
        Sigmode                 & \href{https://github.com/giovannilopez9808/Reconocimiento_de_patrones_proyecto/raw/main/Graphics/Data_2015/PCA_3D_sigmoid.mp4}{Link} & \href{https://github.com/giovannilopez9808/Reconocimiento_de_patrones_proyecto/raw/main/Graphics/Data_2020/PCA_3D_sigmoid.mp4}{Link} \\ \hline
    \end{tabular}
    \caption{Link de descarga para las visualizaciones tridimensionales de los resultados de PCA para el kernel lineal, coseno, gaussiano y sigmoide en los años 2015 y 2020.}
    \label{table:pca_results}
\end{table}

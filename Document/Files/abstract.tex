\begin{center}
    \begin{minipage}{0.8\linewidth}
        \begin{center}
            \changefontsizes{12pt}
            \textbf{Resumen}
        \end{center}
        El índice de marginación busca representar las necesidades básicas insatisfechas utilizando una baja dimensionalidad en los datos. La CONAPO creo este índice en el año 1990 utilizando indicadores para recabar información acerca de la situación socioeconomica de los municipios y entidades. A partir de estos indicadores socioeconomicos utilizan el algoritmo de análisis de componentes principales (PCA) para obtener una representación en una sola dimensión y así clasificar el estado del municipio o entidad. En este trabajo se exploran alternativas dentro de los algoritmos de reducción de dimensionalidad y de clasificación. Resultando así que para casos unidimensionales, el algoritmo PCA con kernel lineal o sigmoide son buenas altertivas para realizar la representación y que los algoritmos de clasificación no son una opción a esta tarea debido a la variedad de resultados.
    \end{minipage}
\end{center}
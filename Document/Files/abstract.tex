\begin{center}
    \begin{minipage}{0.8\linewidth}
        \begin{center}
            \changefontsizes{12pt}
            \textbf{Resumen}
        \end{center}
        El índice de marginación busca representar las necesidades básicas insatisfechas utilizando una baja dimensionalidad en los datos. El Consejo Nacional de Población (CONAPO) creo este índice en el año 1990 empleando indicadores para recabar información acerca de la situación socioeconómica de los municipios y entidades. A partir de estos indicadores socioeconómicos se emplea el algoritmo de análisis de componentes principales (PCA) para obtener una representación en una sola dimensión y así clasificar el estado del municipio o entidad. En este trabajo se exploran alternativas dentro de los algoritmos de reducción de dimensionalidad y de clasificación. Resultando así que para casos unidimensionales, el algoritmo PCA con kernel lineal o sigmoide son buenas alternativas para realizar la representación y que los algoritmos de clasificación no son una opción paraa la tarea debido a la baja presición de resultados.
    \end{minipage}
\end{center}